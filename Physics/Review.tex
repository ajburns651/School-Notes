% Created 2021-05-02 Sun 14:37
% Intended LaTeX compiler: pdflatex
\documentclass[11pt]{article}
\usepackage[utf8]{inputenc}
\usepackage[T1]{fontenc}
\usepackage{graphicx}
\usepackage{grffile}
\usepackage{longtable}
\usepackage{wrapfig}
\usepackage{rotating}
\usepackage[normalem]{ulem}
\usepackage{amsmath}
\usepackage{textcomp}
\usepackage{amssymb}
\usepackage{capt-of}
\usepackage{hyperref}
\setcounter{secnumdepth}{0}
\author{Alex Burns}
\date{\today}
\title{Physics All Subjects notes}
\hypersetup{
 pdfauthor={Alex Burns},
 pdftitle={Physics All Subjects notes},
 pdfkeywords={},
 pdfsubject={},
 pdfcreator={Emacs 27.2 (Org mode 9.5)}, 
 pdflang={English}}
\begin{document}

\maketitle
\setcounter{tocdepth}{1}
\tableofcontents


\section*{Kinematics}
\label{sec:org7a8e614}
\subsection*{Equations}
\label{sec:orgbf2488c}
\(x=\frac12at^2\) \\
\(a=v/t\) \\
\(V^2=v^2+2ax\) \\
\(V=d/t\) \\
\(\Delta d=\frac12(v_f+v_i) \Delta t\) \\
\subsection*{Graphs}
\label{sec:org51c14a5}
\subsection*{Practice Problems}
\label{sec:org90ac3e4}
\begin{itemize}
\item How do we calculate speed? Or velocity?
\item What do distance versus time graphs show? What do the different slopes mean?
\item What is acceleration?
\item When do you use d=st to solve for distance, and when do you have to use x=1/2at2?
\item When do you use s=d/t to solve for speed, and when do you have to use v=at?
\item If we have a curved line when we graph data (like on a distance vs time graph when an object is accelerating), what are the three different ways we can determine speed at a certain point?
\item How can we use the idea of freefall (g=-10m/s2) to solve problems for falling objects?
\item How can we determine the distance a projectile travels if we know the initial conditions of the object?
\item What do we know about the horizontal and vertical components of a projectile as it travels?
\end{itemize}

\section*{Forces}
\label{sec:orgd488dfd}
\subsection*{Big Ideas}
\label{sec:org396ff48}
\begin{itemize}
\item A force is a psuh or a pull on an object
\item Adding up all of the forces on an object will give you the total (\textbf{NET}) force on the object
\item If there is a net force acting on an object, the object will accelerate; if there is no net force the object will maintain a constant (and possibly zero) velocity
\item We can think of forces acting on individual objects or an entire ``system'' of objects. In either case, Newton's 2nd Law: F=ma, still applies
\item Linking forces to other physics topics:
\begin{itemize}
\item Kinematics: Forces cause accelerations or changes in velocity
\item Momentum: Forces cause changes in momentum (impulses)
\item Energy: Forces act on objects to change the kinetic energy through doing work on such objects
\item Oscillations: Forces can oppose motion. When this happens an object will oscillate around a fixed point
\item Rotation: Forces causing rotations act at a distance ``r'' from the point of rotation and are called ``torques''
\end{itemize}
\end{itemize}
\subsection*{What do I really need to remember for the test?}
\label{sec:org1199ff5}
\begin{itemize}
\item You should be able to correctly identify which forces are acting on an object (or a system) at any moment. Remeber, the forces include: gravity, friction, normal, tension, and applied. Generally, air resistance is negligible
\item Make sure you can correctly draw a force diagram. Your force vectors should be drawn neatly and to scale, starting on the object and poitning outward
\item Take a moment to think about which force is larger for a problem without applying any math. Many times students will assume certain relationships that are common, but not always true.  For instance, the normal is often, but not always, equal to gravity.  For a block hangin by a string, tension may or may not be equal to gravity.  Double check the problem carefully!
\item Always start mathematical operations with an equation from the formula sheet.  If you are looking for gravity or friction use Fg=mg and f=μN, respectively.  Solving for centripetal force or the gravitational force between planets, use their respective equations.  If you see a block on a ramp and you say F(II)=mg \(\sin(\theta)\) you will lose points as you have not shown where this equation came from and how it is derived.
\end{itemize}
\subsection*{Equations}
\label{sec:orgfe450df}
\begin{itemize}
\item General
\label{sec:org79ccd0c}
\(F=ma\) \\
\(Fg=mg\) \\
\item Elevator
\label{sec:org99f4580}
\(T-Fg=ma\) \\
\item Ramps
\label{sec:orgfb7fff2}
\(F_N=mg\cos(\theta)\) \\
\(F(II)=mg\sin(\theta)\) \\
\(F_f=uN=umg\cos(\theta)\) \\
\(tan(\theta)=u\) \\
\end{itemize}
\subsection*{Graphs}
\label{sec:org7907e4c}
\subsection*{Practice Problems}
\label{sec:org40210a3}
\begin{itemize}
\item What are Newton’s three laws?
\item If an object has a=0, what else must be true? (There are 4 things)
\item What does it mean when forces are balanced? Or equilibrium?
\item How does the force applied on object A by object B compare to the one applied by object B on object A? (Which law is that?)
\item How does net force relate to acceleration?
\item How is normal force determined?
\item What are some examples of situations in which the normal force is not equal to an object’s weight?
\item When should you consider breaking forces up into component?
\end{itemize}

\section*{Gravity/Circular Motion}
\label{sec:org5205031}
\subsection*{Equations}
\label{sec:org8e69145}
\(F_G=G(m_1m_2)/r^2\) \\
\(F_c=mv^2/r\) \\
\(T^2=(4\pi ^2/GM)r^3\) \\
\subsection*{Graphs}
\label{sec:org7bc4bf7}
\subsection*{Practice Problems}
\label{sec:org26851be}
\begin{itemize}
\item If something is moving in a circle, what must be true about the net force on it? The acceleration?
\item What do we call the net force acting on an object travelling in a circle?
\item How do we calculate Fc? How does it change if we increase the mass?  The velocity? The radius?
\item The Fc is always caused by some force (or sum of forces). For example, for the moon going around the sun, the Fc is caused by gravity. It can be a single force, a component of one force, or the sum of multiple forces. Give an example of each.
\item How does the centripetal force change when a roller coaster is at the top of a loop versus at the bottom of a loop?
\item What caused the centripetal force on the toy airplane in the lab we did? How did we determine the magnitude (value) of the radius and of the centripetal force?
\item How is the force of gravity between two planets affected by their masses? Their distance apart?
\item How is the big equation for gravity (Gmm/r2) related to the weight equation (F=mg)?
\end{itemize}
\section*{Energy}
\label{sec:org1b82133}
\subsection*{Equations}
\label{sec:orgbc4cb09}
\(PE=mgh\) \\
\(KE=\frac12 mv^2\) \\
\(KE(ROT)=\frac12Iw^2\) \\
\(U_s=\frac12kx^2\) \\
\(W=Fx\cos(\theta)\) \\
\subsection*{Graphs}
\label{sec:orgee64a3b}
\subsection*{Practice Problems}
\label{sec:org5794f86}
\begin{itemize}
\item What is “work” in a physics context?  How is it related to the outside force?
\item When does someone do more work or less?
\item How is work related to energy?
\item How do you determine gravitational PE?
\item How do you determine kinetic energy?
\item What does it mean to say that energy is conserved? Or transferred?
\item How are KE and PE related? (How is the height an object begins at related to its speed at h=0?)
\item What is power and how does it relate to energy?
\item How can elastic energy convert into kinetic energy? Or potential?
\end{itemize}
\section*{Momentum}
\label{sec:orgdab7561}
\subsection*{Equations}
\label{sec:org24f08d2}
\(p=mv\) \\
\(\Delta p =m \Delta v\) \\
\(\Delta p = F \Delta t\) \\
\(m_1v_1+m_2v_2=m_1v_1^\prime + m_2v_2^\prime\) \\
\(\frac12 m_1v_1^2+\frac12m_2v_2^2 = \frac12 m_1v_1^2 + m_2v_2^2\) \\
\subsection*{Graphs}
\label{sec:orgc94d761}
\subsection*{Practice Problems}
\label{sec:org215e136}
\begin{itemize}
\item How do you determine momentum?
\item What does it mean to say that momentum is conserved?
\item What is the difference between elastic and inelastic collisions?
\item What is true about the forces on the two objects that are colliding (regardless of mass!)?
\item What is Impulse?
\item How is impulse related to the outside force?
\end{itemize}
\section*{Oscillations \& Waves}
\label{sec:org90907f7}
\subsection*{Equations}
\label{sec:org4ebd0ef}
\(T=2\pi\sqrt{L/g}\) \\
\(T=2\pi\sqrt{m/k}\) \\
\(T=1/f\) \\
\subsection*{Graphs}
\label{sec:org7ff4999}
\subsection*{Practice Problems}
\label{sec:org319c62f}
\begin{itemize}
\item What is an oscillation?  What are some common examples of oscillations?
\item What factors determine the time it takes for things to oscillate (springs and pendulums)
\item What is a wave?  What are some examples of waves?  How are waves organized?
\item Compare and contrast different types of waves.
\item How are the various wave characteristics related?  (ie… period, frequency, wavelength, and wave speed)
\item What are some characteristics of sound waves and light waves?
\item How are open-ended tubes similar and different from closed-ended tubes?
\item How does your eye perceive color?
\item How do colors mix with light as compared to pigments?
\end{itemize}
\section*{Rotation}
\label{sec:orgb5a4325}
\subsection*{Equations}
\label{sec:org693fc35}
\begin{itemize}
\item Torque
\label{sec:org255e28e}
\(\tau=Fr\sin(\theta)\) \\
\(\tau=I\alpha\) \\
\(I=kmR^2\) \\
\item Connecting
\label{sec:org1932310}
\(x=R\theta\) \\
\(v=R\omega\) \\
\(a=R\alpha\) \\
\item General
\label{sec:org930f718}
\(\theta = \frac12 \alpha t^2\) \\
\(\omega_f^2=\omega^2 + 2\alpha \theta\) \\
\(\alpha = \omega / t\) \\
\(\omega = \theta /t\) \\
\end{itemize}
\subsection*{Graphs}
\label{sec:org64e473d}
\subsection*{Practice Problems}
\label{sec:org451f506}
\begin{itemize}
\item How are the rotational motion terms defined?  What is a radian?  How do we measure rotational motion quantities?
\item What is torque in all of its complexity?
\item What is the center of mass of an object?  How can multiple objects have a center of mass? How can you find the center of mass?
\item Why do things balance?  How is balancing related to torque?
\item How does net torque relate to net force?
\item How does the torque on an object relate to the rotational acceleration of an object?
\item How does a pulley work when we include the rotation/torque on the pulley?
\item Why does a ball roll down a hill and a box does not?
\item How does the shape of a round object affect its ability to roll down a hill?  Which shape is the best for rolling down a hill in terms of speed?
\end{itemize}
\end{document}
